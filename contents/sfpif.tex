\chapter{System-Free Picard Integral Formulation}\label{chap:sfpif}

Unlike the broad usage of SSP-RK in various discrete PDE solvers,
the developments of ADER mentioned above have been exclusively applied to FV and DG methods, but FD methods.
This is mainly because the fundamental principle of obtaining high-order accuracy
in the original ADER scheme relies on solving generalized (or high-order) Riemann problems,
which are the characteristic building blocks of FV and DG methods.

Recently, Christlieb et al.\ introduced a new high-order temporal scheme for FDM,
the so-called Picard integral formulation (PIF) method.~\cite{christlieb2015picard,seal2016explicit}
The PIF discretization is based on the constructions of high-order approximation
to the time-averaged fluxes over \( \left[ t^{n}, t^{n + 1} \right] \),
allowing high-order temporal accuracy in a single-step update.
Firstly introduced in~\cite{christlieb2015picard},
the PIF method demonstrated third-order temporally accurate numerical fluxes
by computing the coefficients of the time-Taylor expansion of the averaged fluxes
via LW/CK procedure which converts the high-order temporal derivatives terms into the spatial derivatives.

However, like many Lax-Wendroff type methods, the PIF method requires
finding analytic derivations for flux Jacobians and Hessians.
Although the Jacobian and the Hessian calculations can be easily obtained
with the aid of symbolic manipulators such as \texttt{SymPy}, \texttt{Mathematica}, or \texttt{Maple},
it still demands complicated coding/debugging efforts and ample memory consumption.
Furthermore, as the Jacobian/Hessian calculations highly depend on the
type of the governing system under consideration,
it is required to re-derive the Jacobian/Hessian terms analytically every time
we need to solve a new system, e.g., shallow water equations or magnetohydrodynamics (MHD) equations, to name a few.
In addition, the calculation complexities of the Jacobian-like terms are drastically increasing
with the number of spatial dimensions and the order of accuracy.



\section{Picard Integral Formulation}\label{sec:pif}
Applying the Picard integral formulation (PIF), the governing equations~\cref{eq:gov} can be discretized
by taking a time average within a single time step \( \dt \) over an interval \( \left[ t^{n}, t^{n + 1} \right] \),
\begin{equation}\label{eq:pif_time_avg}
    \bU^{n + 1} = \bU^{n} - \dt \left( \partial_{x} \bF^{avg} + \partial_{y} \bG^{avg} + \partial_{z} \bH^{avg} \right),
\end{equation}
where \( \bF^{avg}, \bG^{avg}, \) and \( \bH^{avg} \) are the time-averaged fluxes in each direction,
\begin{equation}\label{eq:pif_avg_flux}
    \begin{split}
        \bF^{avg} (\bx) = \frac{1}{\dt} \int_{t^{n}}^{t^{n + 1}} \bF (\bU(\bx, t)) \mathop{dt}, \\
        \bG^{avg} (\bx) = \frac{1}{\dt} \int_{t^{n}}^{t^{n + 1}} \bG (\bU(\bx, t)) \mathop{dt}, \\
        \bH^{avg} (\bx) = \frac{1}{\dt} \int_{t^{n}}^{t^{n + 1}} \bH (\bU(\bx, t)) \mathop{dt}, \\
    \end{split}
\end{equation}
for \( \bx = (x, y, z) \in \mathbb{R}^{3} \).

The goal is to express the spatial derivatives of the time-averaged fluxes in~\cref{eq:pif_time_avg}
using highly approximated numerical fluxes \( \hat{\bff}, \hat{\bg}, \) and \( \hat{\bh} \)
at cell interfaces:
\begin{equation}\label{eq:pif_num_flux}
    \begin{split}
        \left. \partial_{x} \bF^{avg} \right|_{\bx = \bx_{ijk}} &=
            \frac{1}{\dx} \left( \hat{\bff}_{\iph, j, k} - \hat{\bff}_{\imh, j, k} \right) + \mathcal{O} (\dx^{p} + \dt^{q}), \\
        \left. \partial_{y} \bG^{avg} \right|_{\bx = \bx_{ijk}} &=
            \frac{1}{\dy} \left( \hat{\bg}_{i, \jph, k} - \hat{\bg}_{i, \jmh, k} \right) + \mathcal{O} (\dy^{p} + \dt^{q}), \\
        \left. \partial_{z} \bH^{avg} \right|_{\bx = \bx_{ijk}} &=
            \frac{1}{\dz} \left( \hat{\bh}_{i, j, \kph} - \hat{\bh}_{i, j, \kmh} \right) + \mathcal{O} (\dz^{p} + \dt^{q}), \\
    \end{split}
\end{equation}
where \( \bx_{ijk} = (x_{i}, y_{j}, z_{k}) \) is the discretization indices.

The above equation is almost analog to~\cref{eq:fdm_flux_deriv},
which is the conventional way to construct numerical fluxes for FDM through
the high-order reconstruction schemes (\cref{eq:fdm_recon}).
The only difference is to take time-averaged fluxes, \( \bF^{avg}, \bG^{avg}, \) and \( \bH^{avg} \)
as an input of the reconstruction scheme rather than taking pointwise fluxes.
Thus~\cref{eq:pif_num_flux} states that with highly approximated time-averaged fluxes,
the resulting numerical fluxes from the conventional reconstruction schemes will be high-order in time and space.

With PIF-numerical fluxes, the governing equation can be expressed in a fully discretized form as,
\begin{equation}\label{eq:pif_full_discrete}
    \begin{split}
        \bU^{n + 1}_{i,j,k} = \bU^{n}_{i,j,k} &- \frac{\dt}{\dx} \left( \hat{\bff}_{\iph,j,k} - \hat{\bff}_{\imh,j,k} \right) \\
                                              &- \frac{\dt}{\dy} \left( \hat{\bg}_{i, \jph, k} - \hat{\bg}_{i, \jmh, k} \right) \\
                                              &- \frac{\dt}{\dz} \left( \hat{\bh}_{i, j, \kph} - \hat{\bh}_{i, j, \kmh} \right),
    \end{split}
\end{equation}
which requires only a single update while attaining high-order accuracy both in time and space.

It is worth remarking that the derived governing form in~\cref{eq:pif_full_discrete} for PIF
is something in between those of FVM and FDM\@. It is different from that of FVM in that
it does not carry any spatial average but the temporal average.
It is also different from that of FDM in that it does involve the temporal average in the fluxes in~\cref{eq:pif_avg_flux},
to which the numerical fluxes \( \hat{\bff}, \hat{\bg}, \) and \( \hat{\bh} \) approximate.

The time-averaged fluxes
are obtained through the Taylor expansion of the pointwise
flux around \( t^{n} \).
In the \( q \)th-order PIF method,
the time-averaged \( x \)-directional flux \( \bF^{avg} \)
is approximated as,
\begin{equation}\label{eq:pif_flux_taylor}
    \begin{split}
        \bF^{avg} (\bx)
        &= \frac{1}{\dt} \int^{t^{n + 1}}_{t^n} \bF(\bx, t) \mathop{dt}\\
        &= \bF (\bx, t^{n})
            + \left. \frac{\Delta t}{2!} \partial_{t}^{(1)} \bF (\bx, t) \right|_{t = t^{n}}
            + \left. \frac{\Delta t^{2}}{3!} \partial_{t}^{(2)} \bF (\bx, t) \right|_{t = t^{n}}
            + \left. \frac{\Delta t^{3}}{4!} \partial_{t}^{(3)} \bF (\bx, t) \right|_{t = t^{n}}
            + \cdots \\
        &= \sum\limits_{i=0}^{q-1}
            \left.\frac{\dt^{i}}{(i+1)!} \partial_{t}^{(i)} \bF (\bx, t) \right|_{t = t^{n}} + \mathcal{O}(\Delta t^{q}) \\
        &= \bF^{appx,q} (\bx, t^{n}) + \mathcal{O}(\Delta t^{q}).
    \end{split}
\end{equation}
The \textit{temporally} \( q \)th-order approximated fluxes \( \bF^{appx, q} \)
will be used
as the inputs of the \( p\)th-order reconstruction scheme \( \mathcal{R}(\cdot) \)
that is combined with a characteristic flux splitting method \( \mathcal{FS}(\cdot) \) 
to apply the \( p \)th-order \textit{spatial} approximation to
the numerical flux \( \hat{\bff} \) at cell interfaces,
\begin{equation}\label{eq:num-flx}
    \hat{\bff}_{i + \half, j, k} =
        \mathcal{R}\left(
            \mathcal{FS}\left(\bF^{appx, q}_{i-r, j}, \dots,
                \bF^{appx, q}_{i+r+1, j}
        \right)
    \right)
    + \mathcal{O}(\dx^{p}),
\end{equation}
where \( r \) represents the stencil radius
required for the \( p \)th-order reconstruction method, \( \mathcal{R}(\cdot) \).
The details of the high-order reconstruction methods, \( \mathcal{R}(\cdot) \), and
the flux-splitting methods, \( \mathcal{FS}(\cdot) \) are described in~\cref{chap:high_order_methods}.

Therefore, the primary objective of the PIF method is to approximate the time-averaged fluxes
in the desired order \( q \), i.e., obtaining \( \bF^{appx, q} \).
For instance, the fourth-order PIF method is characterized by
the fourth-order approximated time-averaged flux in the \( x \)-direction, \( \bF^{appx,4} \),
from the Taylor expansion of the pointwise flux around \( t^{n} \).
As expressed in~\cref{eq:pif_flux_taylor}, the fourth-order PIF method requires
\begin{equation}\label{eq:pif_flux_4order}
\bF^{appx, 4} (\bx)
    = \bF (\bx, t^{n})
        + \left. \frac{\Delta t}{2!} \partial_{t}^{(1)} \bF (\bx, t) \right|_{t = t^{n}}
        + \left. \frac{\Delta t^{2}}{3!} \partial_{t}^{(2)} \bF (\bx, t) \right|_{t = t^{n}}
        + \left. \frac{\Delta t^{3}}{4!} \partial_{t}^{(3)} \bF (\bx, t) \right|_{t = t^{n}}.
\end{equation}
The other \( y \)- and \( z \)-directional approximated fluxes,
\( \bG^{appx, 4} \) and \( \bH^{appx, 4} \),
are defined in similarly.
The only remaining task for the fouth-order PIF method (PIF4) is transforming all the
time derivatives in~\cref{eq:pif_flux_4order} to the corresponding spatial derivatives
via LW/CK procedures;
thereby we could express~\cref{eq:pif_flux_4order} in a fully explicit form.

For simplicity, the compact subscript notation of partial derivatives is adopted
in the following discussions.
The subscripts represent the partial derivatives,
and the temporal expression of \( t = t^{n} \) is omitted.
In the compact notation, \cref{eq:gov} can be rewritten as,
\begin{equation}\label{eq:pif_gov_compact_notation}
    \bU_{t} + \nabla \cdot \mathcal{F}(\bU) = \bU_{t} + \bF_{x} + \bG_{y} + \bH_{z} = 0.
\end{equation}

By applying the chain rule to \( \bF_{t} \),
the evolution equation of the \( x \)-flux, \( \bF \) can be obtained as,
\begin{equation}\label{eq:flux_eqn}
    \bF_{t} = \bF_{\bU} \bU_{t},
\end{equation}
where \( \bF_{\bU} \) is the \( x \)-directional flux Jacobian matrix.
The above equation can be combined with~\cref{eq:pif_gov_compact_notation},
resulting the explicit expression for \( \bF_{t} \) as,
\begin{equation}\label{eq:pif_Ft}
    \bF_{t} = - \bF_{\bU} \Div, \quad \text{where } \Div = \bF_{x} + \bG_{y} + \bH_{z}
\end{equation}
The higher-order time derivatives could be achieved by taking partial derivatives to~\cref{eq:pif_Ft} recursively.
As an example, the second-order term is written as
\begin{equation}\label{eq:pif_Ftt}
    \bF_{tt} = \bF_{\bU \bU} \cdot \Div \cdot \Div - \bF_{\bU} \cdot \Div_{t},
\end{equation}
where
\begin{equation}\label{eq:pif_divt}
    \begin{split}
        \Div_{t} =
            &-\bF_{\bU \bU} \cdot \bU_{x} \cdot \Div - \bF_{\bU} \cdot \Div_{x} \\
            &-\bG_{\bU \bU} \cdot \bU_{y} \cdot \Div - \bG_{\bU} \cdot \Div_{y} \\
            &-\bH_{\bU \bU} \cdot \bU_{z} \cdot \Div - \bH_{\bU} \cdot \Div_{z},
    \end{split}
\end{equation}
and \( \bF_{\bU\bU} \) is the \( x \)-directional flux Hessian tensor.
In Euler equations, the flux Hessians, \( \bF_{\bU\bU}, \bG_{\bU\bU}, \) and \( \bH_{\bU\bU} \)
are the symmetric, rank-3 tensors, so a dot product between the Hessian tensor and a vector
is to be understood as a tensor contraction.
Thus a double dot product between the Hessian tensor and two vectors,
i.e., \( \bF_{\bU \bU} \cdot (\;) \cdot (\;) \) yields a vector of the same dimension with \( \bU \).

Following the same procedure,
an explicit form of the third-order time derivative of the flux
can be obtained as,
\begin{equation}\label{eq:pif_Fttt}
    \bF_{ttt} = -\bF_{\bU \bU \bU} \cdot \Div \cdot \Div \cdot \Div
    + 3 \bF_{\bU \bU} \cdot \Div \cdot \Div_{t}
    - \bF_{\bU} \cdot \Div_{tt},
\end{equation}
where
\begin{equation}\label{eq:pif_divtt}
    \begin{split}
        \Div_{tt} = \quad & \bF_{\bU \bU \bU} \cdot \Div \cdot \bU_{x} \cdot \Div + 2 \bF_{\bU \bU} \cdot \Div \cdot \Div_{x} - \bF_{\bU \bU} \cdot \bU_{x} \cdot \Div_{t} - \bF_{\bU} \cdot \Div_{tx} \\
            + & \bG_{\bU \bU \bU} \cdot \Div \cdot \bU_{y} \cdot \Div + 2 \bG_{\bU \bU} \cdot \Div \cdot \Div_{y} - \bG_{\bU \bU} \cdot \bU_{y} \cdot \Div_{t} - \bG_{\bU} \cdot \Div_{ty} \\
            + & \bH_{\bU \bU \bU} \cdot \Div \cdot \bU_{z} \cdot \Div + 2 \bH_{\bU \bU} \cdot \Div \cdot \Div_{z} - \bH_{\bU \bU} \cdot \bU_{z} \cdot \Div_{t} - \bH_{\bU} \cdot \Div_{tz},
    \end{split}
\end{equation}
and
\begin{equation}\label{eq:pif_divtx}
    \begin{split}
        \Div_{tx} = \quad & \bF_{\bU \bU \bU} \cdot \bU_{x} \cdot \Div \cdot \bU_{x} - \bF_{\bU \bU} \cdot \bU_{xx} \cdot \Div
        -2\bF_{\bU \bU} \cdot \Div_{x} \cdot \bU_{x} - \bF_{\bU} \cdot \Div_{xx} \\
            - & \bG_{\bU \bU \bU} \cdot \bU_{x} \cdot \Div \cdot \bU_{y} - \bG_{\bU \bU} \cdot \bU_{xy} \cdot \Div - \bG_{\bU \bU} \cdot \Div_{x} \cdot \bU_{y} \\
            - & \bG_{\bU \bU} \cdot \bU_{x} \cdot \Div_{y} - \bG_{\bU} \cdot \Div_{xy} \\
            - & \bH_{\bU \bU \bU} \cdot \bU_{x} \cdot \Div \cdot \bU_{z} - \bH_{\bU \bU} \cdot \bU_{xz} \cdot \Div - \bH_{\bU \bU} \cdot \Div_{x} \cdot \bU_{z} \\
            - & \bH_{\bU \bU} \cdot \bU_{x} \cdot \Div_{z} - \bH_{\bU} \cdot \Div_{xz},
    \end{split}
\end{equation}
and similarly for \( \Div_{ty} \) and \( \Div_{tz} \).

Collecting~\crefrange{eq:pif_Ft}{eq:pif_divtx} the fourth-order approximation of the time-averaged \( x \)-flux,
\( \bF^{appx, 4} \) can be expressed in the explicit form, as the spatial derivatives
are readily approximated through the conventional central differencing schemes.
In this dissertation, the conventional five-point central differencing formulae are used:
\begin{equation}\label{eq:pif_central_dfdx}
    \left. \bF_{x} \right|_{\bx = \bx_{ijk}} = \frac{\bF_{i-2} - 8 \bF_{i-1} + 8 \bF_{i+1} - \bF_{i+2} }{12\dx} + \mathcal{O}(\dx^{4}),
\end{equation}
\begin{equation}\label{eq:pif_central_dfdxx}
    \left. \bF_{xx} \right|_{\bx = \bx_{ijk}} = \frac{-\bF_{i-2} + 16 \bF_{i-1} - 30 \bF_{i} + 16 \bF_{i+1} - \bF_{i+2} }{12\dx^{2}} + \mathcal{O}(\dx^{4}).
\end{equation}
For the cross derivatives,
\begin{equation}\label{eq:pif_central_dfdxy}
    \left. \bF_{xy} \right |_{\bx = \bx_{ijk}} =
    \frac{\bF_{i+1, j+1} - \bF_{i-1, j+1} - \bF_{i+1, j-1} + \bF_{i-1 j-1}}{4\dx\dy} + \mathcal{O}(\dx^{2}, \dy^{2}).
\end{equation}

In practical code implementation, reusing the flux divergence \( \Div \)
for calculating high-order spatial derivatives is more efficient than calculating them directly.
For example, \( \Div_{x} \) can be calculated as \( \texttt{dx}(\Div) \),
with the numerical spatial derivative function \( \texttt{dx}(\cdot) \),
rather than calculating as \( \Div_{x} = \bF_{xx} + \bG_{yx}+ \bH_{zx} \).
This approach requires an additional guard cell layer
(resulting in two more guard cells for the five-points derivatives).
However, the overall code performance is better than
evaluating high-order derivatives in each direction without affecting the accuracy of the scheme.

The PIF method is a very efficient numerical strategy to update the solution in FDM formulation.
Once the high-order time-averaged fluxes are determined, the solution can be updated
through a single step by following the exact same process for the conventional FDM spatial reconstruction.
Using the conventional spatial strategy of the FDM formulation,
the PIF method can be ``swapped'' readily with the SSP-RK scheme
in the existing simulation code for improving the code performance.

However, the direct analytic derivations for flux Jacobians, Hessians (and more)
remain as the implementation hurdle for the PIF method.
Unlike SSP-RK methods, the PIF method requires different code implementation
for a different system of equations only because of the \textit{Jacobian-like} terms. (e.g., \( \bF_{\bU}, \bF_{\bU\bU}, \bF_{\bU\bU\bU}, \dots \))
This dissertation aims to tackle this problem,
making a different strategy to use the LW/CK procedure,
which does not require analytical derivations of \textit{Jacobian-like} terms.



\section{System-Free Approach}\label{sec:original_sf}

This section aims to provide a new alternate formulation of computing
the multiplications of Jacobian-vector and Hessian-vector-vector terms in~\crefrange{eq:pif_Ft}{eq:pif_divtx}.
The new approach will replace the necessity for analytical derivations
of the Jacobian-like terms in the original PIF method that is system-dependent,
with a new system-independent formulation, based on the so-called ``Jacobian-free'' method,
which is widely used for Newton-Krylov-type
iterative schemes~\cite{gear1983iterative,brown1990hybrid,knoll2004jacobian,knoll2011application}.

Suppose the Taylor expansion for the flux vector \( \bF \) at a small displacement from \( \bU \),
\begin{subequations}\label{eq:sf_FeV}
    \begin{align}
        \label{eq:sf_FeV_right}\bF ( \bU + \varepsilon \bV) =
        \bF(\bU) + \varepsilon \bF_{\bU} \cdot \bV +
        \frac{1}{2} \varepsilon^{2} \bF_{\bU\bU} \cdot \bV \cdot \bV + \mathcal{O}(\varepsilon^{3}), \\
        \label{eq:sf_FeV_left}\bF ( \bU - \varepsilon \bV) =
        \bF(\bU) - \varepsilon \bF_{\bU} \cdot \bV +
        \frac{1}{2} \varepsilon^{2} \bF_{\bU\bU} \cdot \bV \cdot \bV + \mathcal{O}(\varepsilon^{3}),
    \end{align}
\end{subequations}
where \( \bV \) is an arbitrary vector that has
the same number of components as \( \bU \), and \( \varepsilon \) is a
small scalar perturbation.
By subtracting \cref{eq:sf_FeV_left} from \cref{eq:sf_FeV_right},
we get an expression of a central differencing that is of second-order in $\varepsilon$,
\begin{equation}\label{eq:sf_jac_free}
    \bF_{\bU} \cdot \bV = \frac{1}{2\varepsilon}
    \bigg[ \bF(\bU + \varepsilon \bV) -\bF(\bU - \varepsilon \bV) \bigg]
    + \mathcal{O} ( \varepsilon^{2} ).
\end{equation}
Alternatively, the first-order forward differencing or the backward differencing can be used here.
However, the above second-order central differencing is used for this dissertation,
so that the order of accuracy of the entire system-free approach consistently scales with \( \mathcal{O}(\varepsilon^{2}) \),
given that the Hessian approximation described in the following is to be bounded by \( \mathcal{O}(\varepsilon^{2}) \).
With the system-free approximation of Jacobian, all the Jacobian-vector products in \crefrange{eq:pif_Ft}{eq:pif_divtx}
are to be replaced with the central differencing in \cref{eq:sf_jac_free}.

For the approximation for Hessians, it is imperative to classify
the types of the Hessian tensor contraction.
The first type is the Hessian tensor contracts with the same vector twice, e.g., \( \bF_{\bU\bU} \cdot \bV \cdot \bV \),
and the second type is the tensor contracts with
two different vectors, e.g., \( \bF_{\bU\bU} \cdot \bV \cdot \bW \).

For the first type, we use a Taylor expansion analogous to \cref{eq:sf_FeV}
to approximate the Hessian-vector-vector product with
a central differencing of order \( \mathcal{O}(\varepsilon^{2}) \),
\begin{equation}\label{eq:sf_hes_free_vv}
    \bF_{\bU \bU} \cdot \bV \cdot \bV = \frac{1}{\varepsilon^{2}}
    \bigg[ \bF(\bU + \varepsilon \bV) -2\bF(\bU) -\bF(\bU - \varepsilon \bV) \bigg]
    + \mathcal{O} ( \varepsilon^{2} ).
\end{equation}
Using a simple vector calculus,
the second type can be derived from the first type in \cref{eq:sf_hes_free_vv}
by exploring a symmetric property of the Hessians,
\begin{equation}\label{eq:sf_hes_free_vw}
        \bF_{\bU \bU} \cdot \bV \cdot \bW = \frac{1}{2}
    \bigg[ \bF_{\bU \bU} \cdot \left( \bV + \bW \right) \cdot \left( \bV + \bW \right) -
          \left( \bF_{\bU \bU} \cdot \bV \cdot \bV + \bF_{\bU \bU} \cdot \bW \cdot \bW \right) \bigg].
\end{equation}
The Hessian approximations derived here are now ready to be substituted
in \crefrange{eq:pif_Ftt}{eq:pif_divtx}.

Theoretically speaking, the system-free procedure in the above
can be applied to any arbitrary order of derivatives of the flux function \( \bF \)
with respect to the conservative variable \( \bU \). For instance,
the fourth-order PIF method~\cref{eq:pif_flux_4order}
requires the third-order derivative of \( \bF \), i.e., \( \bF_{\bU\bU\bU} \).
Following the same mathematical basis of \cref{eq:sf_jac_free,eq:sf_hes_free_vv},
the tensor contractions with the same vectors can be approximated as,
\begin{equation}\label{eq:orig-sf-don}
    \begin{split}
        \bF_{\bU \bU \bU} \cdot \bV \cdot \bV \cdot \bV = \frac{1}{2 \varepsilon^{3}}
        \bigg[& -\bF(\bU - 2 \varepsilon \bV) + 2\bF(\bU - \varepsilon \bV) \\
              & - 2\bF(\bU + \varepsilon \bV)+ \bF(\bU + 2 \varepsilon \bV)
        \bigg] + \mathcal{O} ( \varepsilon^{2} ).
    \end{split}
\end{equation}
We can further extend the procedure to compute the contraction
with three different vectors, \( \bV, \bW \), and \( \bX \),
\begin{equation}\label{eq:orig-sf-don-vec}
    \begin{split}
        \bF_{\bU \bU \bU} \cdot \bV \cdot \bW \cdot \bX = \frac{1}{6} \bigg[
            &\bF_{\bU \bU \bU} \cdot \left( \bV + \bW + \bX \right) \cdot \left( \bV + \bW + \bX \right) \cdot \left( \bV + \bW + \bX \right) \\
            & -\bF_{\bU \bU \bU} \cdot \left( \bV + \bW \right) \cdot \left( \bV + \bW \right) \cdot \left( \bV + \bW \right) \\
            & -\bF_{\bU \bU \bU} \cdot \left( \bV + \bX \right) \cdot \left( \bV + \bX \right) \cdot \left( \bV + \bX \right) \\
            & -\bF_{\bU \bU \bU} \cdot \left( \bW + \bX \right) \cdot \left( \bW + \bX \right) \cdot \left( \bW + \bX \right) \\
            & + \bF_{\bU \bU \bU} \cdot \bV \cdot \bV \cdot \bV \\
            & + \bF_{\bU \bU \bU} \cdot \bW \cdot \bW \cdot \bW \\
            & + \bF_{\bU \bU \bU} \cdot \bX \cdot \bX \cdot \bX
        \bigg],
    \end{split}
\end{equation}
and only to see that the number of terms to be computed 
rapidly increases in high-order tensor contraction terms.




\subsection{The proper choices of \( \varepsilon \)}

In the above system-free approximations, the choice of \( \varepsilon \) has to be considered carefully
as it affects the solution accuracy and stability.
On one hand, \( \varepsilon \) is needed to be minimized to improve
the approximated solution accuracy,
the quality of which will scale as the truncation error of \( \mathcal{O}(\varepsilon^{2}) \).
On the other hand, if it is too small the solution would be contaminated
by the floating-point roundoff error which is bounded by
the machine accuracy \( \varepsilon_{\text{mach}} \)~\cite{knoll2004jacobian}.
Therefore, $\varepsilon$ is to be determined judiciously
to provide a good balance between the two types of error.

A recent study by An et al.~\cite{an2011finite}
presents an effective analysis of choosing
\( \varepsilon \) in the context of the Jacobian-free Newton-Krylov iterative framework.
The authors have shown how to compute an ideal value of
\( \varepsilon \) which minimizes the error of the
central differencing in the Jacobian-vector approximation.

The main idea in~\cite{an2011finite} is to find a good
balance between the truncation error \( \mathcal{O}(\varepsilon^{2}) \)
of each Jacobian-free approximation in~\cref{eq:sf_jac_free}
and Hessian-free approximation in~\cref{eq:sf_hes_free_vv},
and the intrinsic floating-point roundoff error $\delta\bF(\bU)$ when calculating
the target exact function value $\bF(\bU)$ with
an approximate value $\bF(\bU) + \delta\bF(\bU)$.
The perturbation $\delta\bF(\bU)$ may include any errors characterized
in computer arithmetic such as roundoff errors, and is assumed to be
bounded by the machine accuracy.

Let $\bF(\bU)$ be an exact function value of $\bF$ at $\bU$, and let
$\bF^{*}(\bU) = \bF(\bU) + \delta \bF(\bU)$ be
an approximation to $\bF(\bU)$,
where $\delta \bF(\bU)$ is a perturbation of
$\bF(\bU)$ that is potentially due from roundoff errors and truncation errors
and is assumed to be bounded by the machine accuracy, i.e.,
$|| \delta\bF(\bU) || \le \varepsilon_{\text{mach}}$.
The main idea is to choose an optimal \( \varepsilon \) value
for the Jacobian-free approximation~\cref{eq:sf_jac_free},
$\varepsilon_{\text{jac}}^{op}$, in such a way that
the error is minimized
when approximating $\bF_{\bU} \cdot \bV$ using
the central differencing approximation of $\bF^{*}(\bU)$
in \cref{eq:sf_jac_free}, i.e.,
\begin{equation}\label{eq:sf_epsilon_jac_free}
    \begin{split}
        \bF_{\bU} \cdot \bV &\approx \frac{1}{2 \sigma} \big[ \bF^{*}(\bU + \sigma \bV) - \bF^{*}(\bU - \sigma \bV) \big] \\
            & = \frac{1}{2 \sigma} \big[ \bF(\bU + \sigma \bV) + \delta \bF(\bU + \sigma \bV)
            -\bF(\bU - \sigma \bV)  - \delta \bF(\bU - \sigma \bV) \big].
    \end{split}
\end{equation}
For the sake of this analysis, we assume that the function
$\bF:\mathbb{R}^n \to \mathbb{R}^n$ is defined to be continuously differentiable sufficiently everywhere,
 $\bF \in C^k(\mathbb{R}^n)$.
We now define the error $E$ by the difference between the central differencing approximation 
in~\cref{eq:sf_epsilon_jac_free} and $\bF_{\bU} \cdot \bV$,
\begin{equation} \label{eq:sf_epsilon_jac_free_error}
    \begin{split}
        E &= \frac{1}{2 \sigma} \big[ \bF^{*}(\bU + \sigma \bV) - \bF^{*}(\bU - \sigma \bV) \big]
            - \bF_{\bU} \cdot \bV \\
          &= \frac{1}{2 \sigma} \big[ \bF(\bU + \sigma \bV) - \bF(\bU - \sigma \bV) \big] +
          \frac{1}{2 \sigma} \big[ \delta \bF(\bU + \sigma \bV) - \delta \bF(\bU - \sigma \bV) \big] -
            \bF_{\bU} \cdot \bV \\
        &= \frac{1}{2 \sigma} \left[
            2 \sigma \bF_{\bU} \cdot \bV +
            \sigma^{3} \int_{0}^{1} \left( 1 - t \right)^{2}  \bF^{(3)} (\bU + t \sigma \bV) \cdot \bV^{3} \mathop{dt}
        \right] \\
        &\qquad + \frac{1}{2 \sigma} \big[ \delta \bF(\bU + \sigma \bV) - \delta \bF(\bU - \sigma \bV) \big] -
            \bF_{\bU} \cdot \bV \\
        &= \mathcal{O}\biggl(\frac{\sigma^2}{2} + \frac{\varepsilon_{\text{mach}}}{2\sigma} \biggr),
    \end{split}
\end{equation}
where the Taylor series expansion around $\bU$ is used for each term
in which the remainders after the third power are given
as the integral form as below,
\begin{equation}\label{eq:sf_epsilon_jac_free_taylor}
    \begin{split}
        \bF(\bU + \sigma \bV) & = \bF(\bU) + \sigma \bF_{\bU} \cdot \bV + \frac{\sigma^{2}}{2} \pdd{\bF}{\bU} \cdot \bV \cdot \bV \\
                              & \qquad + \frac{\sigma^{3}}{2} \int_{0}^{1} \left( 1 - t \right)^{2} \bF^{(3)} (\bU + t \sigma \bV) \cdot \bV^{3} \mathop{dt}, \\
        \bF(\bU - \sigma \bV) & = \bF(\bU) - \sigma \bF_{\bU} \cdot \bV + \frac{\sigma^{2}}{2} \pdd{\bF}{\bU} \cdot \bV \cdot \bV \\
                              & \qquad - \frac{\sigma^{3}}{2} \int_{0}^{1} \left( 1 - t \right)^{2} \bF^{(3)} (\bU + t \sigma \bV) \cdot \bV^{3} \mathop{dt}. \\
    \end{split}
\end{equation}
The optimal choice of \( \varepsilon_{\text{jac}}^{op} \) is to be obtained
by considering the minimization problem of the leading error term in the
last line in~\cref{eq:sf_epsilon_jac_free_error},
\begin{equation}\label{eq:sf_epsilon_jac_optimal}
    \varepsilon^{op}_{\text{jac}} = \argmin_{\sigma > 0}
        \left( \frac{\sigma^{2}}{2} + \frac{\varepsilon_{\text{mach}}}{2 \sigma} \right) =
        {\left( \frac{\varepsilon_{\text{mach}}}{2} \right)}^{\frac{1}{3}} \approx \num{4.8062e-06},
\end{equation}
where \( \varepsilon_{\text{mach}} \sim \num{2.2204E-16} \) is used
assuming a double-precision in a typical 64-bit machine.

Following the similar procedures, the optimal epsilon value for
the Hessian-free approximation~\cref{eq:sf_hes_free_vv}, \( \varepsilon_{\text{hes}}^{op} \)
can be found as,
\begin{equation}\label{eq:sf_epsilon_hes_optimal}
    \varepsilon^{op}_{\text{hes}} = \argmin_{\sigma > 0}
        \left( \frac{\sigma^{2}}{3} + \frac{\varepsilon_{\text{mach}}}{\sigma^{2}} \right) =
        {\left( 3 \varepsilon_{\text{mach}} \right)}^{\frac{1}{4}} \approx \num{1.6065e-04}.
\end{equation}

However, direct use of \( \varepsilon^{op} \) as
the displacement step size in the central differencing schemes
in \cref{eq:sf_jac_free} and \cref{eq:sf_hes_free_vv}
is not a good idea for stability reasons.
%
Usually, the vector \( \bV \) in
\cref{eq:sf_jac_free} and \cref{eq:sf_hes_free_vv} could have an
enormous value in a strong shock region, so it is safer to use
a smaller step size to preserve the needed stability. To meet this,
the ideal value, \( \varepsilon^{op} \) should be normalized
by the magnitude of the vector \( \bV \).
%
There are several prescriptions available
in the Jacobian-free Newton–Krylov 
literatures~\cite{knoll2004jacobian, brown1990hybrid}
to help finalize the decision of choosing a proper value of \( \varepsilon \)
as a function of  \( \varepsilon^{op} \).
%
Nonetheless, as reported in~\cite{lee2021single},
a simple approach of
taking a square root of \( \varepsilon^{op} \)
with a simple normalization is sufficient to attain the desired accuracy and stability,
which is given as,
%
\begin{equation}\label{eq:sf_epsilon_norm}
    \overline{\varepsilon} = \frac{\sqrt{\varepsilon^{op}}}{\left\lVert \bV \right\rVert_{2}}.
\end{equation}

Lastly,
the \( \varepsilon \) estimation can be finalized
by taking the minimum value between $\overline{\varepsilon}$ and $\dt$,
%
\begin{equation}\label{eq:sf_epsilon_min_dt}
    \varepsilon = \min \left( \overline{\varepsilon}, \; \dt  \right),
\end{equation}
to prevent the division by zero case.


\section{Recursive System-Free Approach}\label{sec:recursive_sf}
