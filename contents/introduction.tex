\chapter{Introduction}\label{chap:introduction}

In past decades, the rapid evolution of the high-performance computing (HPC) systems
offers growing computational capacity for the numerical simulations of various scientific fields,
where conducting a direct experiment is extremely expensive or notoriously complicated.
As the computing power of HPC systems gradually increases,
scientists can compute more complex and computationally intensive physical models such as
visualizing black holes~\cite{james2015gravitational,alberdi2019first},
simulating nuclear fusions~\cite{haines2016detailed,gaffney2019making},
studying laser-plasma interactions~\cite{meinecke2014turbulent,tzeferacos2018laboratory},
to name a few.

In order to simulate these physical phenomena,
it demands meticulously designed numerical algorithms for solving
nonlinear, multidimensional, and multi-physics equations judiciously.
Generally speaking, numerical algorithms require more computational power for better solution accuracy,
\textit{i.e.}, using high-resolution grid configuration.

However, the recent hardware development trend
--~the progression of the memory capacity per compute core has become gradually saturated~\cite{attig2011trends}~--
is compelling the HPC community to find more efficient ways
that can best exercise computing resources in pursuing computer simulations.
As reported in 2014~\cite{dongarra2014applied}, decreasing memory density per compute core
will be the primary limiting factor to the scalability of scientific simulation codes.

To meet this end, modern practitioners have relentlessly delved into advancing
high-arithmetic-intensity models that can increase numerical accuracy per degree of freedom
while operating with reduced memory requirements and data transfers in HPC architectures.
For example, in the computational fluid dynamics (CFD) community,
one such computing paradigm is to promote high-order methods in which high arithmetic intensity is achieved
by using an increasing number of higher-order terms.
Due to its high availability in increasing the quality of numerical solutions with fewer grid points,
high-order discrete methods for hyperbolic conservation laws have become primary themes in the CFD community.

Under the dual computational need for accuracy and stability,
the CFD community has developed high-order reconstruction and interpolation strategies
that can achieve spatially high-order
approximation.~\cite{woodward1984numerical,colella1984piecewise,liu1994weighted,jiang1996efficient,borges2008improved,castro2011high,mignone2010high,balsara2016efficient}
However, efforts to achieve a high-order accuracy in the temporal axis have seen a renewed effort.
For decades, multi-stage time integrators~\cite{gottlieb1998total,gottlieb2001strong,gottlieb2011strong}
have been considered as the standard temporal integration strategy
for an extensive range of high-order numerical schemes for partial differential equation (PDE) solvers.

This dissertation develops a single-stage, high-order time integration scheme for hyperbolic PDEs.
The core design concept is to achieve high-order accuracy within a single step
to reduce computational costs for the overall numerical schemes.
Another important objective for designing a high-order time integrator in this dissertation
is to increase its portability.
By designing a time integrator independent of the system of equations,
one can provide increased flexibility and ease of code implementation.

Consequently, a newly developed time integrator showed more than two times faster performance gain
than the conventional multi-stage methods.
Also, it can readily replace only the temporal part of any existing simulation code.
