All the research works presented in this dissertation would not have been possible without the guidance and support of my advisor, Dongwook Lee. Foremostly, I would like to thank Dongwook Lee for his endless support and advice, steering me to the world of numerical methods.

I was very fortunate to participate in the research project at Argonne National Laboratory, which supports the later part of my graduate career. I would like to thank Jeffrey Dooling and Carlo Graziani for giving me an opportunity to join the research project as a visiting graduate student at Argonne National Laboratory.

I also wish to express my gratitude to my Ph.D. committee members, Nicholas Brummell and Hongyun Wang, for giving me many constructive comments and feedback that have improved the dissertation.

Of course, none of what I achieved during my Ph.D. journey would be possible without the love and support of my family, for which I am eternally grateful. 

The text of this dissertation includes reprints of the following previously published material:
\begin{itemize}
    \item Youngjun Lee and Dongwook Lee. A single-step third-order temporal discretization with jacobian-free and hessian-free formulations for finite difference methods. \textit{Journal of Computational Physics}, 427:110063, 2021.
    \item Youngjun Lee, Dongwook Lee, and Adam Reyes. A recursive system-free single-step temporal discretization method for finite difference methods. \textit{Journal of Computational Physics: X}, 12:100098, 2021.
\end{itemize}
The primary co-author Dongwook Lee listed in these publications directed and supervised the research which forms the basis for this dissertation.

I acknowledge the use of the lux supercomputer at UC Santa Cruz, funded by NSF MRI grant AST 1828315.
