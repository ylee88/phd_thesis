\chapter{Discretization Methods}\label{chap:discrete_methods}

This dissertation interests in solving the general conservation laws of hyperbolic PDEs,
predicting numerical solutions with high-order accuracy.
This chapter introduces two general ways to discretize the Euler equations,
which will be of particular interest in this dissertation.

\section{Euler Equations}\label{sec:euler_eqns}

In three dimensions, the conservation laws may be written as,
\begin{equation}\label{eq:gov}
    \partial_{t} \bU + \nabla \cdot \mathcal{F}(\bU)
    = \partial_{t} \bU + \partial_{x} \bF (\bU) + \partial_{y} \bG (\bU) + \partial_{z} \bH (\bU) = 0,
\end{equation}
where \( \bU \) is a vector of conserved variables and
\( \mathcal{F} = {(\bF (\bU), \bG (\bU), \bH (\bU))}^{T} \) is the flux function
in the \( x \), \( y \), and \( z \) directions.
The conservation law is considered hyperbolic
if the flux Jacobian has only real eigenvalues and is diagonalizable. Thus,
\begin{equation}\label{eq:hyperbolic}
    \bA = \partial_{\bU} \bF = \bR \mathbf{\Lambda} \bL,
\end{equation}
where \( \bA \) is the flux Jacobian in \( x \)-direction,
\( \mathbf{\Lambda} \) is a diagonal matrix with real eigenvalues, and
\( \bL \) and \( \bR \) are corresponding left and right eigenvectors.

This dissertation focuses on solving Euler equations, which govern
compressible, adiabatic inviscid flow.
In the Euler equations, the conserved variables and the flux functions are defined as,
\begin{equation}\label{eq:euler-3d}
    \bU = \begin{bmatrix}
        \rho \\
        \rho u \\
        \rho v \\
        \rho w \\
        E
    \end{bmatrix},\;
    \bF (\bU) = \begin{bmatrix}
        \rho u \\
        \rho u^{2} + p \\
        \rho u v \\
        \rho u w \\
        u \left( E + p \right)
    \end{bmatrix},\;
    \bG (\bU) = \begin{bmatrix}
        \rho v \\
        \rho u v \\
        \rho v^{2} + p \\
        \rho v w \\
        v \left( E + p \right)
    \end{bmatrix},\;
    \bH (\bU) = \begin{bmatrix}
        \rho w \\
        \rho u w \\
        \rho v w \\
        \rho w^{2} + p \\
        w \left( E + p \right)
    \end{bmatrix}.
\end{equation}
In the above equations, \( \rho \) is the density,
\( \mathbf{u} = {(u, v, w)}^{T} \) is the velocity,
\( E \) is the total energy, and \( p \) is the pressure of the fluid.
\( E \), the total energy of the fluid represents the
sum of internal and kinetic energy,
\begin{equation}\label{eq:total_E}
    E = \epsilon + \frac{1}{2} \rho \mathbf{u}^{2},
\end{equation}
where the internal energy of the fluid, \( \epsilon \), obeys the equation of state. (EOS)
This dissertation uses the ideal gas law:
\begin{equation}\label{eq:ideal_eos}
    \epsilon = \frac{p}{\gamma - 1},
\end{equation}
where \( \gamma \) is the specific heat ratio.

In the following, this dissertation uses the Euler equation as an example
of the conserved hyperbolic system.
However, numerical methods presented in this dissertation are valid for any system
of the form of~\cref{eq:gov}. Magnetohydrodynamics (MHD) equations, for example,
the general numerical strategies will be nearly identical to the Euler equations
except for the solenoidal constraint of the magnetic field. (\( \nabla \cdot \bB = 0 \))
Thus, the high-order methods presented in this dissertation can be applied to the MHD simulation code easily.


\section{Finite Volume Method}\label{sec:fvm}

The popular way to consider discretized variables for the conserved system
is to cast volume integrals to the governing equations, called the finite volume method. (FVM)
Consider~\cref{eq:gov} discretized on a uniform grid containing cells with equal spacing \( (\dx, \dy, \dz) \)
in the three spatial dimensions.
Then, each cell's center can be indexed by \( (i, j, k) \) at \( (x_{i}, y_{j}, z_{k}) \)
and the cell's face centers at each interface by \( (i \pm \half, j, k), (i, j \pm \half, k), (i, j, k \pm \half) \).
Taking the volume average of each computational cell (\( \frac{1}{\mathcal{V}_{ijk}} \int_{\mathcal{V}_{ijk}} \cdot \mathop{d \mathcal{V}}\))
to~\cref{eq:gov} and applying the divergence theorem, we have,
\begin{equation}\label{eq:fvm}
    \frac{1}{\mathcal{V}_{ijk}} \int_{\mathcal{V}_{ijk}} \partial_{t} \bU \mathop{d \mathcal{V}}
    + \frac{1}{\mathcal{V}_{ijk}} \oint_{\mathcal{S}_{ijk}} \mathcal{F} (\bU) \cdot \mathbf{n} \mathop{d \mathcal{S}} = 0,
\end{equation}
where \( \mathcal{V}_{ijk} \) is the volume of the cell at \( i,j,k \),
and \( \mathcal{S}_{ijk} \) is the surrounding surfaces of the cell at \( i,j,k \).

The semi-discretized form of FVM representation of the conserved system
is obtained by substituting the dimensionally split flux functions
(\( \bF (\bU), \bG (\bU), \bH (\bU) \)):
\begin{equation}\label{eq:fvm_discrete}
    \begin{split}
        \partial_{t} \overline{\bU}_{i,j,k} =& - \frac{1}{\dx} \left( \widetilde{\bF}_{\iph,j,k} - \widetilde{\bF}_{\imh,j,k} \right) \\
                                             & - \frac{1}{\dy} \left( \widetilde{\bG}_{i,\jph,k} - \widetilde{\bG}_{i,\jmh,k} \right) \\
                                             & - \frac{1}{\dz} \left( \widetilde{\bH}_{i,j,\kph} - \widetilde{\bH}_{i,j,\kmh} \right).
    \end{split}
\end{equation}

In the above equation, the overline indicates a volume-averaged quantity,
while the tilde indicates a surface average at half-indexed cell face.
Note that the above semi-discretized form of FVM scheme is a purely analytical result
without any numerical approximation. The numerical methods are used to estimate
surface-averaged fluxes at each cell's interfaces and
update the volume-averaged conserved variables to the next time step.

The most common way to approximate interfacial fluxes for FVM solver for Euler equations is to solve
the Riemann problem at cell interfaces following the Godunov method~\cite{godunov1959difference}.
The Riemann problem is composed of a conservation equation with a single discontinuity
in its initial condition. As firstly introduced by Godunov in~\cite{godunov1959difference},
the Riemann solver (\( \mathcal{RS} \)) gives a numerical flux across the discontinuity in the Riemann problem.
For example, the numerical flux across the discontinuity at \( x_{\iph,j,k} \)
can be calculated as,
\begin{equation}\label{eq:riemann_solver}
    \hat{\bff}_{\iph,j,k} = \mathcal{RS}(\bU^{\text{L}}_{\iph,j,k}, \; \bU^{\text{R}}_{\iph,j,k}).
\end{equation}

The precedent task for solving the Riemann problem is to determining Riemann states at interfaces.
Note that the inputs of the Riemann solver are regarded as \textit{pointwise} values, (\( \bU^{\text{L}}_{\iph,j,k}, \; \bU^{\text{R}}_{\iph,j,k} \))
while the fundamental data type of FVM is the \textit{volume-averaged} values. (\( \overline{\bU}_{i,j,k} \))
To specify the pointwise Riemann states at interfaces with given volume-averaged conserved variables,
they must be reconstructed from the neighboring volume-averaged quantities.
For example, a one-dimensional stencil with radius=\( r \),
the left Riemann states at \( i + \half \) can be reconstructed from
cell-centered volume-averaged conserved variables (\( \overline{\bU}_{i-r,j,k}, \dots, \overline{\bU}_{i,j,k}, \dots, \overline{\bU}_{i+r,j,k} \)),
using \( p \)-th order accurate reconstruction operator \( \mathcal{R}(\cdot) \):
\begin{equation}\label{eq:fvm_recon}
    \bU^{\text{L}}_{\iph,j,k} = \mathcal{R}(\overline{\bU}_{i-r,j,k}, \dots, \overline{\bU}_{i+r,j,k}) + \mathcal{O} (\dx^{p}).
\end{equation}
The spatial order of accuracy, \( p \), of the FVM solver is thereby determined by choice of the reconstruction operator, \( \mathcal{R}(\cdot) \)
which will be discussed in~\cref{sec:recons}.

It is important to note that the numerical flux resulting from the Riemann solver
is also a pointwise representation, while the surface-averaged fluxes
(\( \widetilde{\bF}_{i \pm \half,j,k}, \widetilde{\bG}_{i,j \pm \half,k}, \widetilde{\bH}_{i,j,k \pm \half} \))
are needed for FVM formulation as presented in~\cref{eq:fvm_discrete}.
This should be addressed thoroughly, as a naive approximation of the pointwise flux to
the surface-averaged flux is bounded by second-order accuracy no matter the accuracy of the Riemann states:
\begin{equation}\label{eq:fvm_need_quad}
    \begin{split}
        \widetilde{\bF}_{\iph,j,k} &= \frac{1}{\dy\dz} \int^{y_{\jph}}_{y_{\jmh}} \int^{z_{\kph}}_{z_{\kmh}} \bF(x_{\iph}, y, z) \mathop{dz} \mathop{dy} \\
                                   &= \bF_{\iph,j,k} + \mathcal{O}(\dy^{2}, \dz^{2}).
    \end{split}
\end{equation}
The conventional way to achieve higher than second-order accuracy in FVM solver
is to solve the Riemann problem at multiple quadrature points
on each face.~\cite{titarev2004finite,mccorquodale2011high,zhang2011order}
More recent studies proposed ways to avoid multiple calls of Riemann solver,
reconstructing surface-averaged fluxes from pointwise Riemann fluxes,~\cite{buchmuller2014improved,felker2018fourth}
using linear combinations of Riemann fluxes,~\cite{dumbser2007quadrature,dumbser2008unified}
to name a few.


\section{Finite Difference method}\label{sec:fdm}

As it firstly proposed in~\cite{shu1989efficient}, the finite difference method (FDM) seeks a discretization of
the spatial derivatives of the fluxes in \textit{pointwise} representation.
Assuming that there exist numerical fluxes satisfy the conservative form as,
\begin{equation}\label{eq:fdm_discrete}
    \begin{split}
        \partial_{t} \overline{\bU}_{i,j,k} =& - \frac{1}{\dx} \left( \hat{\bff}_{\iph,j,k} - \hat{\bff}_{\imh,j,k} \right) \\
                                             & - \frac{1}{\dy} \left( \hat{\bg}_{i,\jph,k} - \hat{\bg}_{i,\jmh,k} \right) \\
                                             & - \frac{1}{\dz} \left( \hat{\bh}_{i,j,\kph} - \hat{\bh}_{i,j,\kmh} \right),
    \end{split}
\end{equation}
where \( \hat{\bff}_{i \pm \half,j,k}, \hat{\bg}_{i,j \pm \half,k}, \hat{\bh}_{i,j,k \pm \half} \)
are the \textit{pointwise} numerical fluxes in each direction at half-indexed cell-face centers.
The remaining task is to identify the numerical fluxes in desired order of accuracy, \( p \), which satisfy,
\begin{equation}\label{eq:fdm_flux_deriv}
    \left. \partial_{x} \bF \right|_{\bx = \bx_{ijk}} =
    \frac{1}{\dx} \left( \hat{\bff}_{\iph,j,k} - \hat{\bff}_{\imh,j,k} \right) +
        \mathcal{O} (\dx^{p}), \quad
    \bx_{ijk} = (x_i, y_j, z_k),
\end{equation}
and similarly in \( y \) and \( z \) fluxes.

In order to specify the numerical fluxes for FDM, consider the pointwise \( x \)-flux \( \bF (x, y_{j}, z_{k}) \)
as a one-dimensional cell average of an auxiliary function \( \hat{\bF} \),
\begin{equation}\label{eq:fdm_aux_function}
    \bF(x,y_j,z_k) = \frac{1}{\dx} \int_{x - \frac{\dx}{2}}^{x + \frac{\dx}{2}} \hat{\bF}(\xi, y_j, z_k) \mathop{d\xi}.
\end{equation}
Then the analytic derivative of~\cref{eq:fdm_aux_function} at \( x = x_{i} \) in \( x \)-direction becomes
\begin{equation}\label{eq:fdm_aux_deriv}
    \left. \partial_{x} \bF \right|_{x=x_{i}} =
    \frac{1}{\dx} \left(\hat{\bF}(x_{\iph}, y_j, z_k) - \hat{\bF}(x_{\imh}, y_j, z_k) \right).
\end{equation}
Comparing~\cref{eq:fdm_flux_deriv} and~\cref{eq:fdm_aux_deriv},
the numerical fluxes in FDM are obtained with desired order of accuracy, \( p \),
if they can be defined with
the following relationship with $\hat{\bF}$,
\begin{equation}\label{eq:fdm_num_flux_approx}
    \hat{\bff}_{\iph,j,k} = \hat{\bF} (x_{\iph},y_j,z_k) + \mathcal{O} (\dx^{p}).
\end{equation}
Mathematically speaking, the inverse problem of~\cref{eq:fdm_num_flux_approx} is exactly the same as
the conventional 1D reconstruction problem in FVM, the operation of which is specifically designed to
find the primitive function value $\hat{\bF}$ at a certain location (mostly, $x_{i \pm \half}$) in the $i$-th cell,
given the integral-averaged (or volume-averaged) values ${\bF}$ at nearby stencil points
as input. Namely, this can be written as
\begin{equation}\label{eq:fdm_recon}
    \hat{\bF}(\xi,y_j,z_k) = \mathcal{R}\left(\bF_{i-r,j,k}, \dots, \bF_{i+r,j,k}\right) + \mathcal{O} (\dx^p),
    \quad
    \xi \in [x_{\imh}, x_{\imh}],
\end{equation}
where \( \mathcal{R}(\cdot) \) is a $p$-th order accurate reconstruction operator
that used for FVM formulation in~\cref{sec:fvm}, and will be discussed in~\cref{sec:recons}.

Contrary to FVM, the conservative FDM uses only the pointwise values for constructing numerical strategies,
not requiring the data conversion between pointwise and volume-averaged quantities.
In addition, the high-order reconstruction schemes used for constructing Riemann states in FVM
can be used for constructing numerical fluxes in FDM
without intense changes in the simulation code~--~only a simple change in the input variables for the reconstruction operator.
This simplicity in numerical strategy attracts researchers in the CFD community, leading various adoptions in
high-order solvers.~\cite{jiang1996efficient,shu1998essentially,mignone2010high,christlieb2015picard,seal2016explicit}

Compared to FVM, the major difference of FDM is obtaining high-order numerical fluxes
directly from the reconstruction operator.
Although it simplifies the numerical schemes, the direct formulation of the numerical fluxes
hinders its further modifications while keeping a high-order convergence rate.

For example, the adaptive mesh refinement (AMR) grid configuration~\cite{berger1989local,berger1998adaptive}
requires numerical fluxes splitting between the coarse grid to the fine grid in a conservative manner.
Conventionally, this is ensured by an additional flux correction step in FVM formulation.~\cite{berger1998adaptive}
However, in FDM, a different approach should be taken because modifying the high-order numerical fluxes
may spoil the order of accuracy.
One possible way to maintain conservation across coarse to the fine grid points
is to apply nonlinear interpolation on the conserved variables, imply them as boundary conditions,
and distribute the calculated errors among coarse grid points.~\cite{shen2011adaptive}

Another limitation on the direct formulation of numerical fluxes in FDM is
the lack of the option to include substructure in the wave model,
which Riemann solver typically does in FVM to resolve certain features better.
Del Zanna proposed one alternative way of FDM~\cite{del2003efficient,del2007echo},
which views numerical fluxes as the Riemann fluxes with the series of high-order correction terms.
This approach can achieve a high-order convergence rate with Riemann fluxes like in FVM,
without the additional data type conversions required of conventional FVM.~\cite{reyes2019variable}

\section{Conclusion}\label{sec:discrete_conclusion}

Two major discretization strategies for conservative systems have been presented.
Both the finite difference and finite volume methods are able to achieve
high-order spatial accuracy by reconstructing volume-averaged quantities to the pointwise values.

The finite difference method
\begin{itemize}
    \item evolves the pointwise conserved variables,
    \item provides a straightforward framework without data type conversions,
    \item requires high-order numerical fluxes constructed from the pointwise, cell-centered physical fluxes directly from the high-order reconstruction methods and
    \item may be delicate with the additional modifications on the numerical fluxes.
\end{itemize}

The finite volume method
\begin{itemize}
    \item evolves the volume-averaged conserved variables,
    \item requires rigorous data type conversions to maintain high-order accuracy,
    \item requires high-order reconstruction of the Riemann states from the cell-centered conserved variables and
    \item guarantees the conservation laws over the whole spatiotemporal domain by design.
\end{itemize}
