\chapter{Results}\label{ch:results}

This chapter will provide various numerical test results of SF-PIF methods.
In order to examine the numerical capabilities of the SF method,
several well-known numerical benchmark problems are conducted,
and the traditional SSP-RK methods' results will be provided with the same initial conditions
as counterparts of SF-PIF methods for comparisons.
SF-PIF3 and SF-PIF4 will refer to the \textit{recursive} SF method in~\cref{sec:recursive_sf}
with third-order and fourth-order PIF methods, respectively,
and RK3 and RK4 will refer to the three-stage, third-order SSP-RK method~\cref{eq:ssp_rk3},
five-stage, fourth-order SSP-RK method~\cref{eq:ssp_rk4}, respectively.
The original SF approach (\cref{sec:original_sf}) with the PIF method is denoted by oSF-PIF\@.

\section{Performance of SF-PIF method}\label{sec:result_performance}
The main advantage of the PIF method is the performance gain compared to the SSP-RK methods.
This section will compare the performance of PIF methods (with or without the SF approach)
and the SSP-RK method. The main purpose of this section is to check if the SF-PIF methods provide
improved performance while maintaining the same accuracy as SSP-RK methods.
Theoretically speaking, the SF and oSF approach should not affect the solution's accuracy and stability,
so the original PIF method's results are presented for comparisons.

All test results in this section use the standard fifth-order WENO-JS method (\cref{subsec:weno})
for the spatially high-order reconstruction scheme.
Therefore, the expected truncation error is \( \mathcal{O}(\Delta s^{5}, \dt^{q}) \),
where \( q \) is the order of the temporal scheme.

\section{SF-PIF method with GP-WENO}\label{sec:result_gpweno}

\subsection{Hyperparameters}\label{subsec:result_gp_hyper_params}

\section{SF-PIF method with WENO-JS}\label{sec:result_wenojs}
