\chapter{Conclusion}\label{ch:conclusion}

A new high-order numerical method solving the conservative system of
the partial differential equation has been developed under the finite difference formulation.
The proposed high-order discretization strategy, the SF-PIF method,
has been tested with several benchmark problems
and demonstrated high-order accuracy and fast performance.
The SF-PIF method provides an efficient way to update the solution
in high-order accuracy without significant efforts to implement it into the existing code. 

The third-order in temporal PIF method was originally proposed in~\cite{seal2016explicit},
and it furnishes an efficient single-stage time integration strategy in the finite difference discretization.
Based on the Lax-Wendroff/Cauchy-Kowalevski procedures, the PIF method converts
the time derivatives into spatial derivatives,
ensuring the explicit form of the time-Taylor series.
However, the tight coupling between the spatial and temporal derivatives
requires the flux Jacobian and Hessian tensors.
The flux Jacobians' high-order derivatives are highly dependent on the system of equations;
hence extending the PIF method to higher than fourth-order or implementing it to other systems
is less attractive and requires effort compared to the traditional multi-stage Runge-Kutta methods.

The system-free approach can resolve this issue. The system-free approach provides
a numerical strategy to approximating the tensor contractions
between high-order derivatives of Jacobian tensors and arbitrary vectors
instead of deriving the analytical form of Jacobian-\textit{like} terms.
This new design concept is combined with the original PIF method offering
better portability and extensibility.
The system-free PIF method, SF-PIF method, demonstrated that the numerical approximation
of the tensor contractions does not affect the solution's accuracy and stability.
In the various numerical test results presented in~\cref{sec:result_performance},
the original SF-PIF method produces nearly identical solutions from the PIF method,
maintaining the overall quality of numerical solutions.
As a result, the original SF-PIF method provides an alternative way
to achieve high-order temporal accuracy in the finite difference method,
guaranteeing the same numerical characteristics as the PIF method.

The main concept of designing the system-free approach is to make a numerical method
independent of the system of equations. As described by its name, the system-free method
is entirely independent of the system of equations;
thus, it can be implemented to any other system without a hassle.
The system independence feature of the system-free method is a unique characteristic
among the conventional Lax-Wendroff type schemes that generally require
modifying the code to change the system of equations.
In~\cref{subsec:shallow}, the SF-PIF method demonstrates that
it can be applied to other simulation codes~--~from the Euler equations solver
to the Shallow Water equation solver~--~without changing the numerical part of the code.

However, the original system-free approach demands one additional step
to finalizing the approximation of the tensor contraction,
which makes the system-free approach less extensible to higher than third-order schemes.
The modification step is originated from the discrepancy between
the numerical strategies to approximating tensor contractions of the same vectors and different vectors,
and it requires an exponentially increasing amount of computational costs
in extending to the higher-order scheme.
In this regard, an improved numerical strategy
for the system-free approach is needed for the fourth-order extension of the PIF method.

The recursive version of the system-free method is then proposed.
Instead of directly building approximations for high-order derivatives of the flux Jacobians,
the improved system-free method applies the Jacobian-vector contraction recursively,
avoiding further modifications to handling different vector cases.
Theoretically speaking, the recursive system-free method can approximate rank-4 tensor contractions
more than three times faster than the original system-free method.
Moreover, the recursive system-free method has more compact code structures as well.
Needless to say, the recursive strategy does not degrade the quality of numerical solutions.

The PIF method is a purely high-order temporal scheme,
meaning any spatial high-order reconstruction/interpolation schemes can be combined.
Initially, the PIF method uses the conventional fifth-order WENO reconstruction method,
which is one of the most popular high-order reconstruction methods in the CFD community.
GP-WENO method, on the other hand, is another class of high-order reconstruction scheme proposed recently.
Unlike the conventional reconstruction schemes based on the piecewise polynomials,
the GP-WENO method utilizes the Gaussian process (GP) to estimating the data at unknown points in a stochastic way.
GP-WENO method brings the nonlinear weighting idea from the conventional WENO method
in order to capture the shock discontinuities appropriately,
which is inevitable in nonlinear fluid simulations.
GP-WENO was firstly introduced under the finite volume method and
the primitive variable-based finite difference method,
but the conventional finite difference formulation.
In~\cref{sec:result_gpweno}, several numerical tests demonstrate that
the GP-WENO method provides comparably accurate solutions in the finite difference discretization,
maintaining the fast performance of the SF-PIF methods.
Adopting the GP-WENO into the conventional finite difference formulation demands
additional fine-tuning on the hyperparameters,
which it remained as a further study for this dissertation.

The system-free approach enables the fourth-order single-stage time integration schemes in a better efficient way.
Many high-order research articles in the CFD community designed a ``high-order'' method
by combining mediocre lower-order temporal solver with higher-order spatial solver,
e.g., third-order in temporal, fifth-order spatial in the original PIF method.
It is widely known in the CFD community that the developments of small-scale fluid features
relevant to grid scales depend much more sensitively on the choice of spatial solvers
rather than the temporal solver.
However, as discussed in~\cref{subsec:vortex_weno}, the numerical errors of the temporal solver
dominate the spatial errors on the high-resolution grid configurations,
leading to the degradations of the overall solution accuracy.
This phenomenon should be considered in performing the large-scale simulations
conducted on a high-resolution grid to capture small-scale physics.

The system-free approach is not explicitly designed for the PIF method nor the CFD solvers.
The sole purpose of the system-free method is to approximate the tensor contractions
for high-order derivatives of Jacobians;
thus, it can be applied to any other physics models which need derivations of Jacobians.
Like the SF-PIF methods presented in this dissertation,
the system-free strategy could provide better high-order numerical approximations
in efficiency, flexibility, and portability.
